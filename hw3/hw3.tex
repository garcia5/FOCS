\documentclass[11pt]{article}
\usepackage{color, array, graphics}
\usepackage{amsmath}
\usepackage{amssymb}
\usepackage{enumerate}
\usepackage{mathtools}
\usepackage{fullpage}
\usepackage{graphicx}
\usepackage[utf8]{inputenc}

%Symbol shortcuts
\def\OR{\vee}
\def\AND{\wedge}
\def\imp{\rightarrow}

%Aesthetics
\addtolength{\oddsidemargin}{-.5in}
\addtolength{\evensidemargin}{-.5in}

\addtolength{\topmargin}{-.5in}

\begin{document}

\textbf{Alexander Garcia}

24 February 2017 \\

\begin{enumerate}

	\item Question 1

		\begin{enumerate}[(a)]

			\item $a *_4 -17 = 1$

			$a = 3$\\

			Here, a value of $a=3$ makes $(-17 * a)\ mod\ 4$ a true statement. $-51\ mod\ 4 = 1$ \\


			\item $-17 \div 4 = -5\ rem\ 3$

			This follows from the definition of the modulo operator, where $a\ mod\ b$
			means that $a = bq + r$, and $r$ is always a positive integer. Since $r$ must
			be positive, we obtain the value $-20$ from $bq$, giving us the smallest positive $r$.
			Therefore, $a = 3$. \\

			\item $a \equiv -17\ (mod\ 4)$

			$4 | (a+17)$

			$a = 3$

			$4 | 20$ \\

		\end{enumerate}

	\newpage

	\textbf{Alexander Garcia}	

	24 February 2017 \\

	\item Question 2
		
		$a \equiv b\ (mod\ m)$ 

		\begin{tabular}{ll}
			$m\ |\ a-b$ & Definition of congruency \\

			$\exists c \in \mathbb{Z}, mc = a-b$ & Definition of the ``divisible`` operator \\ 

			$p = gcd(a,m); q = gcd(b,m)$ & Assignment of gcd operations \\

			$\frac{a}{p} = \frac{mc}{p} + \frac{b}{p}$ & Rewriting of ``divisible''
			operator. Here, $\frac{a}{p}$ and $\frac{m}{p}$ are both integers. \\

			$\frac{b}{p} = \frac{-a}{p} + \frac{mc}{p}$ & Reorganization of expression \\

		\end{tabular} \\

		Since the two parts of the equation that make up $\frac{b}{p}$ are both integers,
		$\frac{b}{p}$ must be an integer. Therefore, $p\ |\ b$. In addition, $p \leq q$,
		since $q$ is the largest number that can divide $b$. \\

		\begin{tabular}[]{ll}
			$\frac{a}{q} = \frac{mc}{q} + \frac{b}{q}$ & Rewriting of ``divisible''
			operator. Here, $\frac{mc}{q}$ and $\frac{b}{q}$ are both whole numbers. \\

			
		\end{tabular} \\

		Again, the two parts making up $\frac{a}{q}$ are integers, so $\frac{a}{q}$ must be
		an integer. This time, however, $q \leq p$, for the same reason that $p \leq q$. The
		only possible conclusion then, is that $p &=& q$. \\

		$\therefore gcd(a,m) &=& gcd(b,m)$ \\

		\newpage

		\textbf{Alexander Garcia}

		24 February 2017 \\

	\item Question 3

		\begin{enumerate}[(a)]
		
			\item 
			\begin{tabular}{ll}
				$gcd(124,323) = d, d \in \mathbb{Z}$ & Representation of $gcd$ \\

				$d = sa + tb$ & Bezout's Theorem \\ 

				Now begin stepping through the given algorithm \\

				Representation of gcd & Reformatting of representation \\

				\hline 

				$323 = 2 * 124 + 75$ & $75 = 323 - (2 * 124)$\\

				$124 = 1 * 75 +49$ & $49 = 124 - (1 * 75)$\\

				$75 = 1 * 49 + 26$ & $26 = 75 - *1*49)$\\

				$49 = 1 * 26 + 23$ & $23 = 49 - (1 * 26)$\\

				$26 = 1 * 23 + 3$ & $3 = 26 - (1 * 23)$\\

				$23 = 7 * 3 + 2$ & $2 = 23 - (7 * 3)$\\

				$3 = 1 * 2 + 1$ & $1 = 3 - (1 * 2)$\\

				$2 = 2 * 1$ & $gcd(124, 323) = 1$ \\

			\end{tabular} \\

			We must now go ``backwards'' through these steps and find the coefficients associated with the two numbers ($s$, $t$)
			to make $124s + 323t = 1$ true.

			\begin{tabular}{ll}
				$1 = 3 - (1 * 2)$ & Starting premise\\

				$2 = 23 - (7 * 3)$ & Starting premise\\

				$1 = 3 - 1 * (23 - 7 * 3)$ & $8 * 3 - 1 * 23$ \\

				$8 * (26 - 1 * 23) - 1 * 23$ & $8 * 26 - 9 * 23$ \\

				$8 * 26 - 9 * (49 - 1 * 26)$ & $17 * 26 - 9 * 49$ \\

				$17 * (75 - 1 * 49) - 9 * 49$ & $17 * 75 - 26 * 49$ \\

				$17 * 75 - 26 * (124 - 1 * 75)$ & $43 * 75 - 26 * 124$ \\

				$43 * (323 - 2 * 124) - 26 * 124$ & $43 * 323 - 112 * 124$ \\

			\end{tabular}\\

			The final item in the table contains both of the original numbers,
			and the expression is equal to 1 the whole way down. Therefore, the Bezout
			Coefficients of 124,323 are -112, 43 respectively. \\

			\newpage

			\textbf{Alexander Garcia}

			24 February 2017 \\

			\item This calculation is done through the same steps as part (a).
			
			\begin{tabular}{ll}
				$gcd(3457, 4669) = d, d \in \mathbbm{Z}$ & Representation of $gcd$ \\

				\hline

				$4669 = 1 * 3457 + 1212$ & $1212 = 4669 - 1 * 3457$ \\

				$3457 = 2 * 1212 + 1033$ & $1033 = 3457 - 2 * 1212$ \\

				$1212 = 1 * 1033 + 179$ & $179 = 1212 - 1 * 1033$ \\

				$1033 = 5 * 179 + 138$ & $138 = 1033 - 5 * 179$ \\

				$179 = 1 * 138 + 41$ & $41 = 179 0 1 * 138$ \\

				$138 = 3 * 41 + 15$ & $15 = 138 - 3 * 41$ \\

				$41 = 2 * 15 + 11$ & $11 = 41 - 2 * 15$ \\

				$15 = 1 * 11 + 4$ & $4 = 15 - 1 * 11$ \\

				$11 = 2 * 4 + 3$ & $4 = 11 - 2 * 4$ \\

				$4 = 1 * 3 + 1$ & $1 = 4 - 1 * 3$ \\

				$3 = 3 * 1$ & $gcd(3457, 4669) = 1$ \\
				
			\end{tabular}\\

			Now repeat the steps from before, going backwards to find the Bezout Coefficients.

			\begin{tabular}{ll}

				All expressions in the table are equal to 1. \\

				$1 = 4 - 1 * (11 - 2 * 4)$ & $3 * 4 - 1 * 11$ \\

				$3 * (15 - 1 * 11) - 1 * 11$ & $3 * 15 - 4 * 11$ \\

				$3 * 15 - 4 * (41 - 2 * 15)$ & $11 * 15 - 4 * 41$ \\

				$11 * (138 - 3 * 41) - 4 * 41$ & $11 * 138 - 37 * 41$ \\

				$11 * 138 - 37 * (179 - 1 * 138)$ & $48 * 138 - 37 * 179$ \\

				$48 * (1033 - 5 * 179) - 37 * 179$ & $48 * 1033 - 277 * 179$ \\
				
				$48 * 1033 - 277 * (1212 - 1 * 1033)$ & $325 * 1033 - 277 * 1212$ \\

				$325 * (3457 - 2 * 1212) - 277 * 1212$ & $325 * 3457 - 927 * 1212$ \\

				$325 * 3457 - 927 * (4669 - 1 * 3457) & 1252 * 3457 - 927 * 4669 = 1$ \\

			\end{tabular}

			Bezout Coefficients: $s = 1252, t = -927$ \\ 

		\end{enumerate}

		\newpage

		\textbf{Alexander Garcia}

		24 February 2017 \\

	\item Question 4

		We begin with a proof by contradiction

		Assume there are a finite number of primes of form $q = 3k + 2$ \\

		\begin{tabular}{ll}

			$Q = \{q | q \in \mathbb{P} \cap 3\mathbb{N}+2\}$& Q is the set of ALL primes of the form $3k + 2$ \\

			$N = 3(q_1*q_2*\dots*q_n)+2$ & $3 \nmid N \AND q_i \nmid N, q_i \in Q$ \\ 

		\end{tabular} \\

		It is known that $\nexists q_i \in Q (q_i | N)$, since $q_i | N-2$.
		If $q_i | N$, then $q_i | 2$, which cannot be true, since $q_i$ is an odd prime. \\

		\begin{tabular}{ll}

			$N = odd * even + even \imp N = odd$ & $2 \nmid N$ \\
			
		\end{tabular} \\

		According to the Fundamental Theorem of Arithmtic, any integer (in this case $N$), 
		can be represented uniquely as a product of primes. According to the initial
		assumption, $\forall q \in Q(q \nmid N)$. Then $N$ must be the produt of 
		primes of the form $p_i = 3k + 1$, and would have the form $3k+1$. \\

		\begin{tabular}{ll}

			$N = 3k + 1, k \in \mathbb{Z}$ \\

			However, the premise was that $N$ is of the form $3k + 2$ \\

			$\therefore \exists p, p = 3k + 2 \AND p \not\in Q$ \\

		\end{tabular} \\

		Because $Q$ was defined to be the set of ALL prime numbers of the form 3k+2, we
		have a contradiction.

		$\therefore |Q| = \infty$ \\

		\newpage

		\textbf{Alexander Garcia}

		24 February 2017 \\

	\item Question 5

		\begin{tabular}[]{l l}
			$(A - B) - C$ & \textbf{Premise} \\

			$(A - B) - C = \{x | x \in (A-B) \cap \neg C\}$ &
			\textbf{Definition of set difference} \\

			$A-B = \{x | x \in A \cap \neg B\}$ & 
			\textbf{Definition of set difference} \\

			$(A - B) - C = \{x | x \in A \cap \neg B \cap \neg C\}$ & 
			\textbf{Combination of previous two steps} \\

			$A-C = \{x | x \in A \cap \neg C\}$ & 
			\textbf{Definition of set difference} \\

			$B-C = \{x | x \in B \cap \neg C\}$ & 
			\textbf{Definition of set difference} \\

			$(A-C) - (B-C) = \{x | x \in (A-C) \cap \neg (B-C)\}$ & 
			\textbf{Combination of previous two steps} \\

			$\neg {(B-C)} = \{x | x \in \neg (B \cap \neg C) \}$ &
			\textbf{Negation of set difference} \\

			$\neg (B-C) = \{x | x\in (\neg B \cup C) \}$ & 
			\textbf{DeMorgan's law} \\

			$(A-C)-(B-C) = \{x | x \in (A \cap \neg C) \cap (\neg B \cup C) \}$ &
			\textbf{Combination of previous steps} \\

			$=\{x | x \in (A \cap \neg B \cap \neg C) \cup (A \cap C \cap \neg C) \}$ &
			\textbf{Distributive property} \\

			$=\{x | x \in (A \cap \neg B \cap \neg C) \cup (\emptyset) \}$ &
			\textbf{Complement laws} \\

			$=\{x | x \in A \cap \neg B \cap \neg C \}$ &
			\textbf{Identity laws} \\

		\end{tabular}

		$\therefore (A-B)-C = (A-C) - (B-C)$ \\

		\newpage

		\textbf{Alexander Garcia}

		24 February 2017 \\

	\item Question 6

		\begin{enumerate}[(a)]
				
		\item $A_i = \{i, i + 1, i + 2, \dots\}$

			$\bigcup\limits_{i = 1}^{\infty} A_i = \{1, 2, 3, \dots\}$

			$\bigcap\limits_{i = 1}^{\infty} A_i =$ '$\infty$'  

			Because of the way $A_i$ is defined, the only number that is common amongst all the sets
			is ``the largest number'', which is not exactly a real thing. $A_1 \cap A_2 = [2, \infty)$
			$A_1 \cap A_2 \cap A_3 = [3, \infty)$, and so on. Each time $i$ increments, the final set loses
			the lowest vale, thus just leaving the ``highest number'', represented here as $\infty$.

		\item $A_i = \{0, i\}$ 

			$\bigcup\limits_{i = 1}^{\infty} A_i = \{1, 2, 3, \dots\}$

			$\bigcap\limits_{i = 1}^{\infty} A_i = \emptyset$

		\item $A_i = \{x \in \mathbb{R} | 0 < x < i\}$

			$\bigcup\limits_{i = 1}^{\infty} A_i = \mathbb{R}$

			$\bigcap\limits_{i = 1}^{\infty} A_i = \{x \in \mathbb{R} | 0 < x \leq 1\}$

		\end{enumerate}

\end{enumerate}

\end{document}
