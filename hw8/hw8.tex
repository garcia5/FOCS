\documentclass[11pt]{article}
\usepackage{color, array, graphics}
\usepackage{amsmath}
\usepackage{amssymb}
\usepackage{enumerate}
\usepackage{mathtools}
\usepackage{fullpage}
\usepackage{graphicx}
\usepackage[utf8]{inputenc}
\usepackage{tikz}

%Symbol shortcuts
\def\OR{\vee}
\def\AND{\wedge}
\def\imp{\rightarrow}

%Aesthetics
\addtolength{\oddsidemargin}{-.5in}
\addtolength{\evensidemargin}{-.5in}

\addtolength{\topmargin}{-.5in}

\begin{document}

\textbf{Alexander Garcia}

3 May 2017

	\begin{enumerate}

			\item Question 1

				\begin{enumerate}[(a)]

					\item

						$S \rightarrow 00S1$

						$S \rightarrow 1S00$

						The first two are clear rules of the language. When adding a single 1,
						you must also add two 0's \\

						$S \rightarrow SS$

						Concatenating two existing strings in the grammar will also produce a vaild string.
						Both have the correct relationship between 1's and 0's, so the relationship does not
						change when they are combined \\


						$S \rightarrow 0S1S0$

						The forth rule allows the concatenation of two existing strings, and adding two 0's and a 1,
						allowing for ``random'' placement of 1's and 0's. \\


						$S \rightarrow \lambda$

						The fifth rule allows for the termination of a derivation \\

					\item 	\begin{tabular}{ll}

						$S \rightarrow 00S1$ & Rule 1 \\

						$00S1 \rightarrow 00[1S00]1$ & Rule 2 \\

						$00[1S00]1 \rightarrow 00[1[0S1S0]00]1$ & Rule 4 \\

						$00[1[0S1S0]00]1 \rightarrow 001010001$ & Rule 5 for both $S$ \\

					\end{tabular}

				\end{enumerate}

			\item Question 2

	\end{enumerate}

\end{document}


