\documentclass[]{article}
\usepackage{datetime}
\usepackage{color,array,graphics}
\usepackage{enumerate}
\usepackage{mathtools}
\usepackage{amsmath}
\usepackage{amssymb}
\usepackage{tikz}
\usepackage{tabu}

\addtolength{\oddsidemargin}{-.875in}
\addtolength{\evensidemargin}{-.875in}
\addtolength{\textwidth}{1.75in}

\addtolength{\topmargin}{-.875in}
\addtolength{\textheight}{1.75in}
\voffset0.0in

\def\OR{\vee}
\def\AND{\wedge}
\def\imp{\rightarrow}
\def\math#1{$#1$}
\def\mand#1{$$#1$$}
\def\mld#1{\begin{equation}#1\end{equation}}
\def\eqar#1{\begin{eqnarray}#1\end{eqnarray}}
\def\eqan#1{\begin{eqnarray*}#1\end{eqnarray*}}
\def\cl#1{{\cal #1}}

\DeclareSymbolFont{AMSb}{U}{msb}{m}{n}
\DeclareMathSymbol{\N}{\mathbin}{AMSb}{"4E}
\DeclareMathSymbol{\Z}{\mathbin}{AMSb}{"5A}
\DeclareMathSymbol{\R}{\mathbin}{AMSb}{"52}
\DeclareMathSymbol{\Q}{\mathbin}{AMSb}{"51}
\DeclareMathSymbol{\I}{\mathbin}{AMSb}{"49}
\DeclareMathSymbol{\C}{\mathbin}{AMSb}{"43}


\begin{document}
Alexander Garcia

26 January 2017

661534755

\centerline{\bf \Large Homework 1}

\medskip

\noindent (1) (a) If you miss the final exam, then you do not pass the course. \\

(b) If you have the flu, then you do not pass the course. Or, if you miss the final exam, you do not pass the course. \\

(c) You have the flu and miss the final exam, or you do not miss the final exam and pass the course. \\

\newpage

Alexander Garcia\\

\noindent (2) (a)
\begin{tabular}{c c | c c | c}
	\textbf{P} & \textbf{Q} & $P \imp Q$  & $Q \leftrightarrow \neg P$ & $(P \imp Q) \AND (Q \leftrightarrow \neg P)$\\
	F & T & T & T & T\\
\end{tabular} \\

In this case, $P$ being false will guarantee the first half to be true ($P \imp Q$). Setting $Q = \neg P$ makes the biconditional true, as the biconditional requires each term to be equal. Both of the individual propositons being true makes the compound as a whole true.
\\

(b) This proposition is not satisfiable. Since it is a series of propositions joined by $\AND$, each of the individual pieces much be true in order to make the compound true. However, there are contradicting propositions that make the whole statement impossible. 

$P \OR \neg R$ means that $P$ must be true, or $R$ must be false. In $R \AND S$, $R$ must be true for the statement to be true. In $\neg S \AND \neg P$, $P$ must be false for the statement to be true. Therefore, for the first statement to be true ($P \OR \neg R$), one of the two following statements must be false. Therefore, there is no arrangement that makes the compound proposition true, making it not satisfiable.\\

(c)
\begin{tabular}{c c | c c c c | c}
	\textbf{P} & \textbf{Q} & 
	$P \imp Q$ & $Q \imp P$ & $P \leftrightarrow Q$ & $\neg(P \imp Q)$ & 
	$(P \leftrightarrow Q) \imp \neg(P \imp Q)$ \\
	F & F & T & T & T & F & F \\
	F & T & T & F & F & F & T \\
	T & F & F & T & F & T & T \\
	T & T & T & T & T & F & F \\
\end{tabular} \\

As is clear from the truth table, as long as $P$ and $Q$ are different, the proposition is true. This makes it satisfiable. \\
\newpage

Alexander Garcia\\

\noindent (3)
(a)
$\neg[(\neg p \AND (p \OR q)) \imp q]$

$(\neg p \AND (p \OR q)) \AND \neg q$ \textbf{Logical equivalence for conditional statement}

$\neg p \AND \neg q \AND (p \OR q)$ \textbf{Associative}

$\neg p \AND [(\neg q \AND p) \OR (\neg q \AND q)]$ \textbf{Distributive}

$\neg p \AND [(\neg q \AND p) \OR F]$ \textbf{Negation}

$\neg p \AND (\neg q \AND p)$ \textbf{Identity}

$(\neg p \AND \neg q) \AND (\neg p \AND p)$ \textbf{Distributive}

$(\neg p \AND \neg q) \AND F$ \textbf{Negation}

$F$ \textbf{Domination}\\

(b) $\neg [((\neg p \OR q) \AND (\neg q \OR r)) \imp (p \imp r)]$

$((\neg p \OR q) \AND (\neg q \OR r)) \AND \neg(p \imp r)$ \textbf{Logical equivalence for conditional statement}

$((\neg p \OR q) \AND (\neg q \OR r)) \AND (p \AND \neg r)$ \textbf{Logical equivalence for conditional statement}

$p \AND (\neg p \OR q) \AND (\neg q \OR r) \AND \neg r$ \textbf{Associative}

$[(p \AND \neg p) \OR (p \AND q)] \AND (\neg q \OR r) \AND \neg r$ \textbf{Distributive}

$[F \OR (p \AND q)] \AND (\neg q \OR r) \AND \neg r$ \textbf{Negation}

$(p \AND q) \AND (\neg q \OR r) \AND \neg r$ \textbf{Domination}

$(p \AND q) \AND [(\neg r \AND \neg q) \OR (\neg r \AND r)]$ \textbf{Distributive}

$(p \AND q) \AND [(\neg r \AND \neg q) \OR F]$ \textbf{Negation}

$(p \AND q) \AND (\neg r \AND \neg q)$ \textbf{Domination}

$p \AND \neg r \AND q \AND \neg q$ \textbf{Associative}

$p \AND \neg r \AND F$ \textbf{Negation}

$F$ \textbf{Domination}

\newpage

Alexander Garcia\\

\noindent (4) (a) False. Solving the basic equation $x + 3 = 10$ gives the result $x = 7$. Since 7 is not in the domain of $x$, the statement is false. \\

(b) True. Given that the domain of $y$ is limited to $\{1, 2, 3\}$, $y+ 3$ cannot be greater than 7. \\

(c) True. The statement says that there is at least one $x$ which when squared, is less than every $y + 1$. A value of 1 for $x$ is less than $y + 1$ for every $y$ in the domain. \\

(d) False. Since the domain of $x$ is $\{1, 2, 3, 4, 5\}$, the statement is false for every $x \geq 4$ in the set, no matter the $y$ chosen.

\newpage

Alexander Garcia\\

\noindent (5) (a) False. $5 = 5 / 5; (x, y = 5)$ \\

(b) True.

$y^4 - x < 16$

$y ^ 4 - 16 < x$

Choose $y = 1$

$-15 < x$

Because the domain of $x$ is the set of all integers $(\geq 0)$, $x$ cannot be smaller than the result.\\

(c) True. Because of the qualifiers, the statement means that both curves over the domain of all integers are different, or they never touch. At $x, y = 0$, $log_2(x) = -\infty$, and $y ^ 3 = 0$. At $(x, y = 1)$, $log_2(x) = 0$, and $y^3 = 1$. So from $0 \leq x, y \leq 1$, the two curves do not touch. After this point, the $y ^ 3$ curve grows at a rate of about $y ^ 2$. The $log_2(x)$ curve grows at a rate of about $\frac{1}{x}$. Because the curve that is lower at the start grows at a slower rate than the higher curve, it will never touch the higher curve.

\end{document}