\documentclass[11pt]{article}
\usepackage{color, array, graphics}
\usepackage{enumerate}
\usepackage{mathtools}
\usepackage{fullpage}
\usepackage[utf8]{inputenc}
\usepackage{amsmath}
\usepackage{amsthm}
\usepackage{eucal}
\usepackage{amssymb}
\usepackage{mathrsfs}

%Symbol shortcuts
\def\OR{\vee}
\def\AND{\wedge}
\def\imp{\rightarrow}

%Aesthetics
\addtolength{\oddsidemargin}{-.5in}
\addtolength{\evensidemargin}{-.5in}

\addtolength{\topmargin}{-.5in}


\begin{document}

\textbf{Alexander Garcia}

24 March 2017 \\
\begin{enumerate}

	\item Question 1

		\begin{enumerate}[(a)]

			\item $\sum^{n}_{i=1}(2i-1)$

				\begin{tabular}{ll}
					$n = 1$ & $\sum^{n}_{i=1} (2i-1) = 1$ \\

					$n = 2$ & $\sum^{n}_{i=1} (2i-1) = 1 + 3 = 4$ \\

					$n = 3$ & $\sum^{n}_{i=1} (2i-1) = 4 + 5 = 9$ \\

					$n = 4$ & $\sum^{n}_{i=1} (2i-1) = 9 + 7 = 16$ \\
				\end{tabular}

				A possible formula for this summation would be
				$f(n) = n^2, n \geq 1$ \\

			\item Proof by mathematical induction

				$n^2 = \sum^{n}_{i=1} (2i-1)$

				Base case: $1^2 = \sum^{1}_{i=1} (2i-1)$

				$1 = 2-1$ \\

				Assume:

				$n^2 = \sum^{n}_{i=1} (2i-1)$

				$(n+1)^2 = \sum^{n+1}_{i=1} (2i-1)$

				$\sum^{n+1}_{i=1} (2i-1) = \sum^{n}_{i=1} (2i-1) + 2(n+1)-1$

				$(n+1)^2 = n^2 + 2n + 1$

				$n^2 + 2n + 1 = \sum^{n}_{i=1} (2i-1) + 2n + 1$ \\

				Subtract $(2n+1)$ from each side

				$n^2 = \sum^{n}_{i=1} (2i-1)$ \\

				It is assumed that $n^2 = \sum^{n}_{i=1} (2i-1)$ from the
				inductive step. Therefore, the formula is correct.\\

		\end{enumerate}

		\newpage

		\textbf{Alexander Garcia}

		24 March 2017 \\

	\item Question 2

\end{enumerate}

\end{document}


