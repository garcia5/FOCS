\documentclass[11pt]{article}
\usepackage{color, array, graphics}
\usepackage{enumerate}
\usepackage{mathtools}
\usepackage{fullpage}
\usepackage[utf8]{inputenc}
\usepackage{amsmath}
\usepackage{amsthm}
\usepackage{eucal}
\usepackage{amssymb}
\usepackage{mathrsfs}

%Symbol shortcuts
\def\OR{\vee}
\def\AND{\wedge}
\def\imp{\rightarrow}

%Aesthetics
\addtolength{\oddsidemargin}{-.5in}
\addtolength{\evensidemargin}{-.5in}

\addtolength{\topmargin}{-.5in}


\begin{document}

\textbf{Alexander Garcia}

24 March 2017 \\
\begin{enumerate}

	\item Question 1

		\begin{enumerate}[(a)]

			\item $\sum^{n}_{i=1}(2i-1)$

				\begin{tabular}{ll}
					$n = 1$ & $\sum^{n}_{i=1} (2i-1) = 1$ \\

					$n = 2$ & $\sum^{n}_{i=1} (2i-1) = 1 + 3 = 4$ \\

					$n = 3$ & $\sum^{n}_{i=1} (2i-1) = 4 + 5 = 9$ \\

					$n = 4$ & $\sum^{n}_{i=1} (2i-1) = 9 + 7 = 16$ \\
				\end{tabular}

				A possible formula for this summation would be
				$f(n) = n^2, n \geq 1$ \\

			\item Proof by mathematical induction

				$n^2 = \sum^{n}_{i=1} (2i-1)$

				Base case: $1^2 = \sum^{1}_{i=1} (2i-1)$

				$1 = 2-1$ \\

				Assume:

				$n^2 = \sum^{n}_{i=1} (2i-1)$

				$(n+1)^2 = \sum^{n+1}_{i=1} (2i-1)$

				$\sum^{n+1}_{i=1} (2i-1) = \sum^{n}_{i=1} (2i-1) + 2(n+1)-1$

				$(n+1)^2 = n^2 + 2n + 1$

				$n^2 + 2n + 1 = \sum^{n}_{i=1} (2i-1) + 2n + 1$ \\

				Subtract $(2n+1)$ from each side

				$n^2 = \sum^{n}_{i=1} (2i-1)$ \\

				It is assumed that $n^2 = \sum^{n}_{i=1} (2i-1)$ from the
				inductive step. Therefore, the formula is correct.\\

		\end{enumerate}

		\newpage

		\textbf{Alexander Garcia}

		24 March 2017 \\

	\item Question 2

		Proof by mathematical induction

		Base Case: $P(3) := \frac{3(2)(1)}{6} = 1$

		This is true, since a set of size 3 can only have
		one subset of size 3. \\

		Inductive Step: Assume a set $S$ of size $n$ has
		$\frac{n(n-1)(n-2)}{6}$ subsets of size 3. Also
		it is given that $S$ has $\frac{n(n-1)}{2}$ subsets
		of size 2. \\

		$R = S + \{x_{n+1}\}$

		$|R| = n+1$

		$S \subset R$

		By adding the new element $\{x_{n+1}\}$, $R$ now
		has a certain number of subsets of size 3 that
		contain $\{x_{n+1}\}$, and a certain number that do
		not.

		The number of subsets of size 3 without the new
		element is simply the number of size 3 subsets in
		$S$.

		$S$ has $\frac{n(n-1)}{2}$ subsets of size 2, as
		given by the problem.

		The number of subsets of size 3 that include the
		new element is also $\frac{n(n-1)}{2}$. This is due
		to the fact that $S$ and $R$ are identical, save
		the new element. By adding the new element to every
		size 2 subset in $S$, one can generate every size
		3 subset containing $\{x_{n+1}\}$ in $R$.

		Thus, the number of size 3 subsets in $R$ is
		$\frac{n(n-1)(n-2)}{6} + \frac{n(n-1)}{2} =
		\frac{n(n+1)(n-1)}{6}$

		$P(n+1):=\frac{n(n+1)(n-1)}{6}$

		$\therefore P(n) \rightarrow P(n+1)$

		\newpage

		\textbf{Alexander Garcia}

		29 March 2017 \\

	\item Question 3

		Proof by Strong Induction

		Base Cases:

		$n = 1: 1 \leq 2$

		$n = 2: 2 \leq 4$

		$n = 3: 3 \leq 8$ \\

		Inductive Hypothesis:

		$P(n):= a_{n-1}+a_{n-2}+a_{n-3} \leq 2^n$ \\

		$a_{n+1} = a_n + a_{n-1} + a_{n-2}$

		$a_n = a_{n-1} + a_{n-2} + a_{n-3}$

		Substitute in $a_n$

		$a_{n+1} = 2a_{n-1} + 2a_{n-2} + a_{n-3}$

		$2a_{n-1} + 2a_{n-2} + a_{n-3} <
		2(a_{n-1} + a_{n-2} + a_{n-3})$

		$2^{n+1} = 2^n * 2$

		We just showed that $a_{n+1} < 2*a_n$

		$2*a_n \leq 2*2^n$, since by the inductive
		hypothesis, $a_n \leq 2^n$

		So

		$a_{n+1} < 2*a_n \leq 2*2^n$

		Simplifying this down,

		$a_{n+1} \leq 2^{n+1} = P(n+1)$

		$\therefore P(n) \rightarrow P(n+1)$

		\newpage

		\textbf{Alexander Garcia}

		29 March 2017 \\

	\item Question 4

		Proof by Strong Induction

		Base Case:

		$20 = 5r + 6s$

		$r = 4, s = 0$ \\

		Inductive Hypothesis:

		$\forall n \in \mathbb{Z}_{>20}\ n = 5r + 6s$ \\

		$21 = 5(3) + 6(1)$

		$22 = 5(2) + 6(2)$

		$23 = 5(1) + 6(3)$

		$24 = 5(0) + 6(4)$

		We now let $r_n$ represent the $r$ coefficient
		used in the construction of $n$, and $b_n$
		represent the $b$ coefficient of $n$

		$25 = 5(5) + 6(0)$

		$25 = 5(r_{20}+1) + 6(b_{20})$

		$n = 5(r_n) + 6(b_n)$

		$n+5 = 5(r_n + 1) + 6(b_n)$

		By increasing a number by 5, the only change that
		needs to be made to its $r$ value is to increase
		it by 1. The $b$ value can remain the same.
		Therefore, $P(n) \rightarrow P(n+5)$

		By proving the first 4 base cases, we then have
		proven $P(n) \forall n \geq 20$.

		\newpage

		\textbf{Alexander Garcia}

		29 March 2017 \\

	\item Question 5

		\begin{enumerate}[(a)]

			\item

				Basis Step:

				For $T$ of size 1, $H(T) = 0$

				Recursive Step:

				For $T = T_1 \bullet T_2 \bullet
				T_3$,

				$H(T) =
				max(H(T_1), H(T_2), H(T_3)) + 1$ \\

			\item

				Basis Step:

				For $T$ that is a single vertex,
				$N(T) = 1$

				Recursive Step:

				For $T = T_1 \bullet T_2 \bullet
				T_3$,

				$N(T)=1+N(T_1) + N(T_2) + N(T_3)$\\

			\item

		Base Case:

		For $T$ of one vertex $r$

		$N(T) = 1$

		$H(T) = 0$

		$3^{H(T)+1}-1 = 2$

		$1<2$ \\

		Inductive Hypothesis:

		For a full ternary tree $T$,

		$N(T) \leq 3^{H(T)+1}-1$ \\

		Recursive Step:

		$T = T_1 \bullet T_2 \bullet T_3$

		$N(T) = 1 + N(T_1) + N(T_2) + N(T_3)$

		$N(T) \leq 1 + (3^{H(T_1)+1}-1) + (3^{H(T_2)+1}-1)
		+ (3^{H(T_3)+1}-1)$, by the inducive hypothesis

		$N(T) \leq 3 * max(3^{H(T_1)+1}, 3^{H(T_2)+1}, 3^{H(T_3)+1}) - 2$ \\

		This is true since the sum of these three terms cannot be
		more than $3*$ the largest term. \\

		$max(3^{H(T_1)+1}, 3^{H(T_2)+1}, 3^{H(T_3)+1}) =
		3^{max(H(T_1), H(T_2), H(T_3)) + 1}$

		$3^{max(H(T_1), H(T_2), H(T_3)) + 1} = H(T)$

		Substituting this into the earlier equation gives us

		$N(T) \leq 3 * 3^{H(T)} - 2$

		$N(T) \leq 3^{H(T)+1} - 2 < 3^{H(T)+1} - 1$

		\end{enumerate}

\end{enumerate}

\end{document}


