\documentclass[11pt]{article}
\usepackage{color, array, graphics}
\usepackage{amsmath}
\usepackage{amssymb}
\usepackage{enumerate}
\usepackage{mathtools}
\usepackage{fullpage}
\usepackage{graphicx}
\usepackage[utf8]{inputenc}

%Symbol shortcuts
\def\OR{\vee}
\def\AND{\wedge}
\def\imp{\rightarrow}
\DeclarePairedDelimiter\ceil{\lceil}{\rceil}
\DeclarePairedDelimiter\floor{\lfloor}{\rfloor}

%Aesthetics
\addtolength{\oddsidemargin}{-.5in}
\addtolength{\evensidemargin}{-.5in}

\addtolength{\topmargin}{-.5in}

\begin{document}

\textbf{Alexander Garcia}

9 March 2017

\begin{enumerate}

	\item Question 1

		\begin{enumerate}[(a)]

			\item $f(x) = -3x^2 + 7$

				$f(-1) = 4$

				$f(1) = 4$ \\

				Obviously, $1 \in \mathbb{R} \AND -1 \in \mathbb{R}$, so both are within
				the domain of $f$. This goes against the definition of a bijection, since
				two different elements in the domain of $f$ have the same image. In order
				to rectify this, the domain of $f$ should be $\{x \in \mathbb{R} : x \geq 0\}$.
				The range would also have to be modified to be $\{y \in \mathbb{R} : y \geq 7\}$,
				since this function will never be less than 7. \\

				Inverse: $x = -3y^2 + 7$

				$\frac{7-x}{3} = y^2$

				$y = \sqrt{\frac{7-x}{3}}$

				$f^{-1}(x) = \sqrt{\frac{7-x}{3}}$

				This is, of course, given the modified domain and range. \\

			\item $f(x) =  \frac{x+2}{x+2}$

				$f(-2) = DNE$ \\

				In order for $f$ to be a bijection, every element in its domain must have
				an image in its range. Since $f(-2)$ is undefined, the conditions are not
				satisfied. Here, the domain of $f$ could be modified to be
				$\{x \in \mathbb{R} : x \neq -2\}$. The range could be modified to be
				$\{y \in \mathbb{R} : y \neq 1\}$, since this function, by definition,
				can never be equal to 1. \\

				Inverse: $x = \frac{y+1}{y+2}$

				$x * (y + 2) = y + 1$

				$xy + 2x = y + 1$

				$2x - 1 = y - xy$

				$2x - 1 = y(1 - x)$

				$f^{-1}(x) = \frac{2x-1}{1-x}$ \\

			\item $f(x) = x^5 + 1$

				This function is a bijection, since every element in the domain has exactly one unique
				image.

				Inverse: $x = y^5 + 1$

				$x-1 = y^5$

				$f^{-1}(x) = \sqrt[5]{x-1}$
				$\{y \in \mathbb{R}\}$
				$\{x \in \mathbb{R}\}$


		\end{enumerate}

	\textbf{Alexander Garcia}

	9 March 2017 \\

	\item Question 2

		$f(x) = ax + b$

		$g(x) = cx + d$

		$\{a, b, c, d \in \mathbb{R}\}$

		$f \circ g = a(cx + d) + b$

		$g \circ f = c(ax + b) + d$

		$a(cx + d) + b = c(ax + b) + d$

		$acx + ad + b = cax + cb + d$

		$ad + b = cb + d$

		$(f \circ g = g \circ f) \leftrightarrow
		(ad + b = cb + d)$ \\

		%I think that's it? double check

		\newpage

		\textbf{Alexander Garcia}

		9 March 2017 \\

	\item Question 3

		Proof by Cases:

		In each case, $x = n + q$

		Case 1: $0 \leq q < \frac{1}{3}$

		\begin{tabular}{ll}
			$3x = 3n + 3q$ \\

			$\floor*{3x} = 3n$ &
			Because $0 \leq 3n < 1$ \\

			$\floor*{x + \frac{1}{3}} = n$ &
			$x+\frac{1}{3} = n + \frac{1}{3} + q$ and
			$0 \leq \frac{1}{3} +q < 1$\\

			$\floor*{x + \frac{2}{3}} = n$ &
			$x+\frac{2}{3} = n + \frac{2}{3} + q$ and
			$0 \leq \frac{1}{3} + q < 1$\\

			$\floor*{x} = n$ \\

		\end{tabular}

		$\floor*{x} + \floor*{x + \frac{1}{3}} +
		\floor*{x + \frac{2}{3}} = n + n + n = 3n =
		\floor*{3x}$ \\

		Case 2: $\frac{1}{3} \leq q < \frac{2}{3}$

		\begin{tabular}{ll}
			$3x = 3n + 3q$ & $3x = (3n + 1) + (3q-1)$\\

			$\floor*{3x} = 3n + 1$ & $0 \leq 3q -1<1$\\

			$\floor*{x + \frac{1}{3}} = n$ &
			$x + \frac{1}{3} = n + \frac{1}{3} + q$ and
			$0 \leq q - \frac{1}{3} < 1$ \\

			$\floor*{x + \frac{2}{3} }=
			\floor*{n + \frac{2}{3} + q}$ \\

			$\floor*{x + \frac{2}{3}} = n + 1$ \\

		\end{tabular}

		$\floor*{x} + \floor*{x + \frac{1}{3} } +
		\floor*{x + \frac{2}{3} } = n + n + n + 1 = 3n + 1
		= \floor*{3x}$ \\

		Case 3: $\frac{2}{3} \leq q < 1$

		\begin{tabular}{ll}
			$3x = 3n + 3q$ & $3x = (3n + 2)+(3q - 2)$\\

			$\floor*{3x} = 3n + 2$ & $0 \leq 3q-2 < 1$\\

			$\floor*{x + \frac{1}{3}} = n + 1$ &
			$x + \frac{1}{3} = n+1+ (q-\frac{2}{3})$ and
			$0 \leq q - \frac{2}{3} < 1$ \\


			$\floor*{x + \frac{2}{3}} = n + 2$ &
			$x + \frac{2}{3} = n+1+ (q-\frac{1}{3})$ and
			$0 \leq q - \frac{2}{3} < 1$ \\

		\end{tabular}

		$\floor*{x} + \floor*{x + \frac{1}{3}} +
		\floor*{x + \frac{2}{3}} = n + n + 1 + n + 1 =
		3n + 2 = \floor*{3x}$

		Since all three cases satisfy the initial statement,
		it is correct. \\

		\newpage

		\textbf{Alexander Garcia}

		9 March 2017 \\

	\item Question 4

		\begin{enumerate}[(a)]

			\item $a_n = n^5$

				$a_0 = 0$

				$a_1 = 1$

				$a_2 = 32$

				$a_3 = 243$ \\

			\item $a_n = n^2 + n$

				$a_0 = 0$

				$a_1 = 2$

				$a_2 = 6$

				$a_3 = 12$

				$a_4 = 20$

				As this sequence continues, it is clear that the difference between
				each term increases by two each time. The recurrence relation can
				then be seen to be

				$$a_n = a_{n-1} + 2^n$$

				with the initial condition being that $a_0 = 0$\\

			\item $a_n = n + (-1)^n$

				$a_0 = 1$

				$a_1 = 0$

				$a_2 = 3$

				$a_3 = 2$

				$a_4 = 5$

				$a_5 = 4$

				This sequence simply switches each pair of items. Every even number
				becomes the next odd number, and every odd number becomes the
				previous even number. The recurrence relation is then very
				straightforward.

				$$a_n = a_{n-2} + 2$$

				With the initial conditions being $a_0 = 1, a_1 = 0$ \\

		\end{enumerate}

	\newpage

	\textbf{Alexander Garcia}

	9 March 2017

	\item Question 5

		\begin{enumerate}[(a)]

			\item $P(2):= 2! < 2^2$

			\item $2*1 < 2 * 2$

				$2 < 4$

			\item $\exists k \in \mathbb{Z}_{\geq 0}: P(k)$

			\item $P(k) \rightarrow P(k+1)$

			\item $P(k+1):= (k+1)! < (k+1)^{(k+1)}$

				Assume $P(k)$ is true.

				$(k+1)! = k! * (k+1)$

				$(k+1)! < k^k * (k+1)$

				This is true by definition of the inductive hypothesis.

				$k^k < (k+1)^k$

				Due to the inequality, we can substitute in this term
				to the previous relationship of $(k+1)!$

				$(k+1)! < (k+1)^k * (k+1)$

				$(k+1)! < (k+1)^{(k+1)}$

			\item This inequality is true only when $n > 1$, because the base
				case would not be true for $n = 1$. If the base case does
				not hold, then the inductive reasoning cannot be used to
				prove that it is true $\forall n$. However, since
				$P(k) \rightarrow P(k+1)$, and we know that $P(2)$ holds,
				then we know that $P(2) \rightarrow P(3) \rightarrow P(4)$
				and so on.

		\end{enumerate}

	\newpage

	\textbf{Alexander Garcia}

	9 March 2017

	\item Question 6

		Proof by induction

		$P(n):= 3|n^3+2n$, when $n$ is a positive integer.

		Base case: $2^3 + 2*2 = 12$

		$3 | 12$

		Inductive Hypothesis: $P(k):= 3|k^3 + 2k$, when $k$
		is a positive integer.

		Prove $P(k) \rightarrow P(k+2)$, since $k$ must be a
		positive integer

		$(k+2)^3 + 2(k+2) = (k^3 + 6k^2 + 12k + 8) + (2k + 4)$

		$=(k^3 + 2k) + (6k^2 + 12k + 12)$

		$=(k^3 + 2k) + 3(2k^2 + 4k + 4)$

		$3|k^3 + 2k$ by the inductive hypothesis

		$3 | 3(2k^2 + 4k + 4)$ because by definition, $n | an$

		$\therefore 3 | (k+1)^3 + 2(k+1)$, proving the
		initial statement.

\end{enumerate}

\end{document}


