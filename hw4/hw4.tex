\documentclass[11pt]{article}
\usepackage{color, array, graphics}
\usepackage{amsmath}
\usepackage{amssymb}
\usepackage{enumerate}
\usepackage{mathtools}
\usepackage{fullpage}
\usepackage{graphicx}
\usepackage[utf8]{inputenc}

%Symbol shortcuts
\def\OR{\vee}
\def\AND{\wedge}
\def\imp{\rightarrow}

%Aesthetics
\addtolength{\oddsidemargin}{-.5in}
\addtolength{\evensidemargin}{-.5in}

\addtolength{\topmargin}{-.5in}

\begin{document}

\textbf{Alexander Garcia}

9 March 2017

\begin{enumerate}

	\item Question 1

		\begin{enumerate}[(a)]

			\item $f(x) = -3x^2 + 7$

				$f(-1) = 4$

				$f(1) = 4$ \\

				Obviously, $1 \in \mathbb{R} \AND -1 \in \mathbb{R}$, so both are within
				the domain of $f$. This goes against the definition of a bijection, since
				two different elements in the domain of $f$ have the same image. In order
				to rectify this, the domain of $f$ should be $\{x \in \mathbb{R} : x \geq 0\}$.
				The range would also have to be modified to be $\{y \in \mathbb{R} : y \geq 7\}$,
				since this function will never be less than 7. \\

				Inverse: $x = -3y^2 + 7$

				$\frac{7-x}{3} = y^2$

				$y = \sqrt{\frac{7-x}{3}}$

				$f^{-1}(x) = \sqrt{\frac{7-x}{3}}$

				This is, of course, given the modified domain and range. \\

			\item $f(x) =  \frac{x+2}{x+2}$

				$f(-2) = DNE$ \\

				In order for $f$ to be a bijection, every element in its domain must have
				an image in its range. Since $f(-2)$ is undefined, the conditions are not
				satisfied. Here, the domain of $f$ could be modified to be
				$\{x \in \mathbb{R} : x \neq -2\}$. The range could be modified to be
				$\{y \in \mathbb{R} : y \neq 1\}$, since this function, by definition,
				can never be equal to 1. \\

				Inverse: $x = \frac{y+1}{y+2}$

				$x * (y + 2) = y + 1$

				$xy + 2x = y + 1$

				$2x - 1 = y - xy$

				$2x - 1 = y(1 - x)$

				$f^{-1}(x) = \frac{2x-1}{1-x}$ \\

			\item $f(x) = x^5 + 1$

				This function is a bijection, since every element in the domain has exactly one unique
				image.

				Inverse: $x = y^5 + 1$

				$x-1 = y^5$

				$f^{-1}(x) = \sqrt[5]{x-1}$
				$\{y \in \mathbb{R}\}$
				$\{x \in \mathbb{R}\}$


		\end{enumerate}

	\item Question 2

		$f(x) = ax + b$

		$g(x) = cx + d$

		$\{a, b, c, d \in \mathbb{R}\}$

		$f \circ g = a(cx + d) + b$

		$g \circ f = c(ax + b) + d$

		$a(cx + d) + b = c(ax + b) + d$

		$acx + ad + b = cax + cb + d$

		$ad + b = cb + d$

		$(f \circ g = g \circ f) \leftrightarrow
		(ad + b = cb + d)$ \\

		%I think that's it? double check

\end{enumerate}

\end{document}


