\documentclass[11pt]{article}
\usepackage{color, array, graphics}
\usepackage{amsmath}
\usepackage{amssymb}
\usepackage{enumerate}
\usepackage{mathtools}
\usepackage{fullpage}
\usepackage{graphicx}
\usepackage[utf8]{inputenc}

%Symbol shortcuts
\def\OR{\vee}
\def\AND{\wedge}
\def\imp{\rightarrow}

%Aesthetics
\addtolength{\oddsidemargin}{-.5in}
\addtolength{\evensidemargin}{-.5in}

\addtolength{\topmargin}{-.5in}

\begin{document}

\textbf{Alexander Garcia}

6 April 2017 \\

\begin{enumerate}

	\item Question 1

		\begin{enumerate}[(a)]

			\item Let $\mathbb{S}$ be the set of odd multiples of 3

				Basis step: $3 \in \mathbb{S}$

				Recursive step: If $x \in \mathbb{S} \AND y \in \mathbb{S}$

				$x+2y \in \mathbb{S}$

				$x-2y \in \mathbb{S}$

				Because $x, y \in \mathbb{S}$, both are known to be odd. It is also
				known that $even * odd = even$, and that $even + odd = odd$. Each of the
				terms is also a multiple of 3, and multiplying that value by an integer
				does not change that fact. Therefore, an odd multiple of 3 $\pm$ an even multiple
				of 3 generates a new odd multiple of 3. \\

			\item Let $\mathbb{S}$ be the set of bit strings with an even number of zeros

				Note: This question is done assuming that a string of zero length
				has an even number of zeros.

				Basis Step: $\emptyset \in \mathbb{S}$

				Recursive Step: $x \in \mathbb{S}$

				$1x \in \mathbb{S}$

				$x1 \in \mathbb{S}$

				$00x, 0x0, x00 \in \mathbb{S}$

				Adding a 1 to the string will not affect the number of zeros, so the string
				is still valid. Adding two zeros in any location will then maintain an
				even number of zeros. \\

			\item Let $\mathbb{S}$ be the set of strings of even length
				from the alphabet $\Sigma = \{a, b\}$

				Basis Step: $\emptyset \in \mathbb{S}$

				Recursive Step: $X \in \mathbb{S}$

				$Xaa, Xab, Xba, Xbb \in \mathbb{S}$

		\end{enumerate}

	\newpage

	\textbf{Alexander Garcia}

	6 April 2017 \\

	\item Question 2

		\begin{enumerate}[(a)]

			\item The maximum number of members is the product of the
				possible choices for each character in the membership
				number.

				Total capital letters: $26$

				Total number of length 5 bit strings: $2^5 = 32$

				Total number of 2 digit numers where
				the second digit is less than the first:

				$(1+2+3+4+5+6+7+8+9) = 45$

				$26 * 32 * 45 = 37440$ possible unique member numbers\\

			\item In this case, the lowest possible number of shared first letters
				would be $\frac{1000}{26} $. This is because all numbers must
				start with a capital letter. Any one letter not being shared
				means the rest must be shared by more people. So, the
				lowest number for any given letter would be in an even distribution,
				which is represented by $\frac{1000}{26} $. This means that there
				are at least 38 people with the same first letter in a group of 1000. \\

			\item The total number of possible numbers that don't start with ``J'' is

				$25 * 32 * 45 = 36000$

				This answer means that there are more than enough possible numbers
				for 2000 members that do not begin with ``J''. So, in a group of
				2000, there are at least 0 membership numbers that begin wtih ``J''.

		\end{enumerate}

		\newpage

		\textbf{Alexander Garcia}

		6 April 2017 \\

	\item Question 3

		\newpage

		\textbf{Alexander Garcia}

		6 April 2017 \\

	\item Question 4

		\begin{enumerate}[(a)]

			\item

			\item

		\end{enumerate}

		\newpage

		\textbf{Alexander Garcia}

		6 April 2017 \\

	\item Question 5

		\begin{enumerate}[(a)]

			\item

			\item

		\end{enumerate}

		\newpage

		\textbf{Alexander Garcia}

		6 April 2017 \\

	\item Question 6

		\begin{enumerate}[(a)]

			\item

			\item

			\item

		\end{enumerate}

\end{enumerate}

\end{document}


