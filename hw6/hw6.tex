\documentclass[11pt]{article}
\usepackage{color, array, graphics}
\usepackage{amsmath}
\usepackage{amssymb}
\usepackage{enumerate}
\usepackage{mathtools}
\usepackage{fullpage}
\usepackage{graphicx}
\usepackage[utf8]{inputenc}

%Symbol shortcuts
\def\OR{\vee}
\def\AND{\wedge}
\def\imp{\rightarrow}

%Aesthetics
\addtolength{\oddsidemargin}{-.5in}
\addtolength{\evensidemargin}{-.5in}

\addtolength{\topmargin}{-.5in}

\begin{document}

\textbf{Alexander Garcia}

6 April 2017 \\

\begin{enumerate}

	\item Question 1

		\begin{enumerate}[(a)]

			\item Let $\mathbb{S}$ be the set of odd multiples of 3

				Basis step: $3 \in \mathbb{S}$

				Recursive step: If $x \in \mathbb{S} \AND y \in \mathbb{S}$

				$x+2y \in \mathbb{S}$

				$x-2y \in \mathbb{S}$

				Because $x, y \in \mathbb{S}$, both are known to be odd. It is also
				known that $even * odd = even$, and that $even + odd = odd$. Each of the
				terms is also a multiple of 3, and multiplying that value by an integer
				does not change that fact. Therefore, an odd multiple of 3 $\pm$ an even multiple
				of 3 generates a new odd multiple of 3. \\

			\item Let $\mathbb{S}$ be the set of bit strings with an even number of zeros

				Note: This question is done assuming that a string of zero length
				has an even number of zeros.

				Basis Step: $\emptyset \in \mathbb{S}$

				Recursive Step: $x \in \mathbb{S}$

				$1x \in \mathbb{S}$

				$x1 \in \mathbb{S}$

				$00x, 0x0, x00 \in \mathbb{S}$

				Adding a 1 to the string will not affect the number of zeros, so the string
				is still valid. Adding two zeros in any location will then maintain an
				even number of zeros. \\

			\item Let $\mathbb{S}$ be the set of strings of even length
				from the alphabet $\Sigma = \{a, b\}$

				Basis Step: $\emptyset \in \mathbb{S}$

				Recursive Step: $X \in \mathbb{S}$

				$Xaa, Xab, Xba, Xbb \in \mathbb{S}$

		\end{enumerate}

	\newpage

	\textbf{Alexander Garcia}

	6 April 2017 \\

	\item Question 2

		\begin{enumerate}[(a)]

			\item The maximum number of members is the product of the
				possible choices for each character in the membership
				number.

				Total capital letters: $26$

				Total number of length 5 bit strings: $2^5 = 32$

				Total number of 2 digit numers where
				the second digit is less than the first:

				$(1+2+3+4+5+6+7+8+9) = 45$

				$26 * 32 * 45 = 37440$ possible unique member numbers\\

			\item In this case, the lowest possible number of shared first letters
				would be $\frac{1000}{26} $. This is because all numbers must
				start with a capital letter. Any one letter not being shared
				means the rest must be shared by more people. So, the
				lowest number for any given letter would be in an even distribution,
				which is represented by $\frac{1000}{26} $. This means that there
				are at least 38 people with the same first letter in a group of 1000. \\

			\item The total number of possible numbers that don't start with ``J'' is

				$25 * 32 * 45 = 36000$

				This answer means that there are more than enough possible numbers
				for 2000 members that do not begin with ``J''. So, in a group of
				2000, there are at least 0 membership numbers that begin wtih ``J''.

		\end{enumerate}

		\newpage

		\textbf{Alexander Garcia}

		6 April 2017 \\

	\item Question 3

		Take an arbitrary vertex $V_1$

		Because the graph is complete, $V_1$ shares an edge (i.e. has some
		relationship with) all of the eight other vertices.

		It is given that in any complete, two color, 9 vertex graph, there is
		at least one node incident to 6 red edges, or 4 blue edges. We will
		then assume that this is the case for $V_1$

		We then consider the set of 6 friends of $V_1$, from the previous
		statement. This ``sub graph'' can be treated as its own 6 vertex
		party, which had been covered previously.

		For a 6 vertex graph, it is known that there exists a 3-clique of either
		friends or enemies.

		If there is a 3 clique of friends, and all members of the clique are also
		friends with $V_1$, then we have found a 4-clique of friends (red edges)

		If there is a 3 clique of enemies, then we have found the 3-clique of enemies
		we were looking for in the beginning.

		$\therefore$ Any complete 2-colored graph with 9 vertices contains either a
		red 4-clique or a blue 3-clique \\

		\newpage

		\textbf{Alexander Garcia}

		6 April 2017 \\

	\item Question 4

		\begin{enumerate}[(a)]

			\item A 10 bit string with fewer ones than zeros has either
				6, 7, 8, 9, or 10 zeros.

				The number of bit strings with exactly 6 zeros is
				$C(10, 6) = \frac{10!}{6!(10-6)!}=210$

				With 7 zeros:
				$C(10, 7) = \frac{10!}{7!(10-7)!}=120$

				With 8 zeros:
				$C(10, 8) = \frac{10!}{8!(10-8)!}=45$

				Wtih 9 zeros:
				$C(10, 9) = \frac{10!}{9!(10-9)!}=10$

				Wtih 10 zeros:
				$C(10, 10) = 1$

				The sum of these is the total number of bit strings of
				length 10 with fewer ones than zeros

				$210+120+45+10+1 = 386$ \\

			\item We can determine the number of length 10 bit strings with
				at least 5 consecutive ones or zeros by treating each group of 5
				as a single item. This changes the number of positions from
				10 to 6.

				When the 5 bit sequence starts at the beginning of the string, there
				are $2^5$ possible strings. However, because the rest of the string
				could contain more zeros, we must avoid counting some combinations twice.

				This means that the sequence being in positions 2-6 only have $2^4$ new
				possible strings associated with them.

				The same principle applies for a sequence of ones. The only remaining overlap
				is for the strings 0000011111 and 1111100000, as they are counted by both
				the sequences of 5 ones and 5 zeros.

				Therefore, the total number of length 10 bit strings is
				$2*(2^5 + 5*2^4) - 2 = 222$

		\end{enumerate}

		\newpage

		\textbf{Alexander Garcia}

		6 April 2017 \\

	\item Question 5

		\begin{enumerate}[(a)]

			\item Treat the problem as $w+x+y+z+u = 100$

				The number of solutions to this problem is
				the same as the number of ways one can select
				$n_1$ of w, $n_2$ of x, $n_3$, of y, $n_4$ of z,
				and $n_5$ of u, where $n_1+n_2+n_3+n_4+n_5 = 100$

				This is equal to the number of 100 combinations from
				a set of 5 elements.

				$\binom{5+100-1}{100} = \binom{104}{100}$

				According to Theorem 2 from the text, $\binom{n+r-1}{r}=
				\binom{n+r-1}{n-1}$

				So, the number of solutions for this equation is
				$\binom{104}{4} = \frac{104!}{4!(100!)} = 4598126$ \\

			\item Again, treat the problem as $w+x+y+z+u = 100$

				This time, $x>1$ or $y>1$

		\end{enumerate}

		\newpage

		\textbf{Alexander Garcia}

		6 April 2017 \\

	\item Question 6

		\begin{enumerate}[(a)]

			\item

			\item

			\item

		\end{enumerate}

\end{enumerate}

\end{document}


