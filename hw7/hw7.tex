\documentclass[11pt]{article}
\usepackage{color, array, graphics}
\usepackage{enumerate}
\usepackage{mathtools}
\usepackage{fullpage}
\usepackage[utf8]{inputenc}

%Symbol shortcuts
\def\OR{\vee}
\def\AND{\wedge}
\def\imp{\rightarrow}

%Aesthetics
\addtolength{\oddsidemargin}{-.5in}
\addtolength{\evensidemargin}{-.5in}

\addtolength{\topmargin}{-.5in}


\begin{document}

\textbf{Alexander Garcia}

21 April 2017 \\

	\begin{enumerate}

		\item Question 1

			\begin{enumerate}[(a)]

				\item When rolling the Zocchihedron the first time, you are guaranteed to get a number.
					The probability of rolling the same number in any of the next four rolls is
					$\frac{1}{100}$. The sum of these probabilities is $\frac{4}{100} = \frac{1}{25} = 0.04$\\

				\item A fair die will have exactly $\frac{1}{100}$ chance of rolling a given number.
					Using the same logic as in part (a), we must solve for the number of rolls, rather
					than the probability.

					$\frac{n}{100} = 0.3$

					$n = 30$ rolls \\


			\end{enumerate}

\newpage

\textbf{Alexander Garcia}

21 April 2017 \\

		\item Question 2

			\begin{enumerate}[(a)]

				\item

				\item

			\end{enumerate}
\newpage
\textbf{Alexander Garcia}

21 April 2017 \\

		\item Question 3

			\begin{enumerate}[(a)]

				\item

				\item

				\item

			\end{enumerate}
\newpage
\textbf{Alexander Garcia}

21 April 2017 \\

		\item Question 4
\newpage
\textbf{Alexander Garcia}

21 April 2017 \\

		\item Question 5
\newpage
\textbf{Alexander Garcia}

21 April 2017 \\

		\item Question 6

			\begin{enumerate}[(a)]

				\item

				\item

			\end{enumerate}

	\end{enumerate}

\end{document}


