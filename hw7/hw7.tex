\documentclass[11pt]{article}
\usepackage{color, array, graphics}
\usepackage{enumerate}
\usepackage{mathtools}
\usepackage{fullpage}
\usepackage[utf8]{inputenc}
\usepackage{amsmath}

%Symbol shortcuts
\def\OR{\vee}
\def\AND{\wedge}
\def\imp{\rightarrow}

%Aesthetics
\addtolength{\oddsidemargin}{-.5in}
\addtolength{\evensidemargin}{-.5in}

\addtolength{\topmargin}{-.5in}


\begin{document}

\textbf{Alexander Garcia}

21 April 2017 \\

	\begin{enumerate}

		\item Question 1

			\begin{enumerate}[(a)]

				\item When rolling the Zocchihedron the first time, you are guaranteed to get a number.
					The probability of rolling the same number in any of the next four rolls is
					$\frac{1}{100}$. The sum of these probabilities is $\frac{4}{100} = \frac{1}{25} = 0.04$\\

				\item A fair die will have exactly $\frac{1}{100}$ chance of rolling a given number.
					Using the same logic as in part (a), we must solve for the number of rolls, rather
					than the probability.

					$\frac{n}{100} = 0.3$

					$n = 30$ rolls \\


			\end{enumerate}

\newpage

\textbf{Alexander Garcia}

21 April 2017 \\

		\item Question 2

			\begin{enumerate}[(a)]

				\item Probability Distribution: $\sum_{s \in S}{p(s)} = 1$

					$p(100) = 5p(n), n \in \{1, 2, \cdots, 99\}$

					$p(1) = p(2) = \cdots = p(99)$

					$99*p(1) + 5*p(1) = 1$

					$104*p(1) = 1$ \\

					$p(i) = \frac{1}{104}, i \in \{1, 2, \cdots, 99\}$

					$p(100) = \frac{5}{104}$ \\

				\item For any one roll, the $p(100) = \frac{5}{104}$. Thus, the
					probability of rolling a 100 in any n rolls is $\frac{5n}{104}$, since
					the probability each time compounds, adding another $\frac{5}{104}$

			\end{enumerate}
\newpage
\textbf{Alexander Garcia}

21 April 2017 \\

		\item Question 3

			\begin{enumerate}[(a)]

				\item In order for a bit string of length $n$ to contain an equal number of 1's and 0's,
					it must have $\frac{n}{2}$ of each (1, 0 are treated as heads and tails over n flips).
					For a bit string of length $n$ there are $C(n, \frac{n}{2})$ strings with equal 1's and
					0's.

					There are $2^n$ total possible outcomes of flipping the coin $n$ times, and
					$C(n, \frac{n}{2})$ of these will have an even number of heads and tails.

					The probability is then $\frac{\frac{n!}{(n/2)!(n/2)!}}{2^n} $. \\

				\item It will not affect the original probability.

					Limiting the first flip to heads limits the number of total
					possible outcomes to $2^{n-1}$.

					Of the remaining flips, there must be exactly one more tail result than heads.
					This number of strings is $C(n-1, \lceil{\frac{n}{2} -1}\rceil)$, where
					$\lceil{\frac{n}{2} -1}\rceil = \frac{n-1}{2}$, since $n$ must be odd.

					Probability = $\frac{\frac{(n-1)!}{(\frac{n-1}{2})!(\frac{n-1}{2})!}}{2^{n-1}}
					* \frac{\frac{n}{n/2}}{2} = \frac{\frac{n!}{(n/2)!(n/2)!}}{2^n} $

					Since the ratios are the same, the probabilities are the same. \\

				\item $E(X) = \sum_{t \in S}{p(t)X(t)}$

					$X(t) = \sum_{i = 1}^{n}X_i(t)$

					$E(X) = \sum_{i = 1}^{n}{(\sum_{j = 1}^{i}{0.5 * 2} + \sum_{j = 1}^{i}{0.5 * -1})}$

					$=\sum_{i = 1}^{n}{i - \frac{i}{2} } = \sum_{i = 1}^{n}{\frac{i}{2} }$ \\

					The expected value is the sum of all the possibilities that the random variable $X(T)$ can take,
					weighted by their probabilities. There is a $0.5$ probability that a flip will come up heads, and the
					same for tails. Here, $n$ would be the maximum number of flips, and $i$ is the number of flips for
					the current trial.

			\end{enumerate}
\newpage
\textbf{Alexander Garcia}

21 April 2017 \\

		\item Question 4
\newpage
\textbf{Alexander Garcia}

21 April 2017 \\

		\item Question 5
\newpage
\textbf{Alexander Garcia}

21 April 2017 \\

		\item Question 6

			\begin{enumerate}[(a)]

				\item

				\item

			\end{enumerate}

	\end{enumerate}

\end{document}


